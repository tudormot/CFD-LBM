\documentclass[]{article}
\usepackage{lmodern}
\usepackage{amssymb,amsmath}
\usepackage{ifxetex,ifluatex}
\usepackage{fixltx2e} % provides \textsubscript
\ifnum 0\ifxetex 1\fi\ifluatex 1\fi=0 % if pdftex
  \usepackage[T1]{fontenc}
  \usepackage[utf8]{inputenc}
\else % if luatex or xelatex
  \ifxetex
    \usepackage{mathspec}
  \else
    \usepackage{fontspec}
  \fi
  \defaultfontfeatures{Ligatures=TeX,Scale=MatchLowercase}
\fi
% use upquote if available, for straight quotes in verbatim environments
\IfFileExists{upquote.sty}{\usepackage{upquote}}{}
% use microtype if available
\IfFileExists{microtype.sty}{%
\usepackage{microtype}
\UseMicrotypeSet[protrusion]{basicmath} % disable protrusion for tt fonts
}{}
\usepackage{hyperref}
\hypersetup{unicode=true,
            pdfborder={0 0 0},
            breaklinks=true}
\urlstyle{same}  % don't use monospace font for urls
\IfFileExists{parskip.sty}{%
\usepackage{parskip}
}{% else
\setlength{\parindent}{0pt}
\setlength{\parskip}{6pt plus 2pt minus 1pt}
}
\setlength{\emergencystretch}{3em}  % prevent overfull lines
\providecommand{\tightlist}{%
  \setlength{\itemsep}{0pt}\setlength{\parskip}{0pt}}
\setcounter{secnumdepth}{0}
% Redefines (sub)paragraphs to behave more like sections
\ifx\paragraph\undefined\else
\let\oldparagraph\paragraph
\renewcommand{\paragraph}[1]{\oldparagraph{#1}\mbox{}}
\fi
\ifx\subparagraph\undefined\else
\let\oldsubparagraph\subparagraph
\renewcommand{\subparagraph}[1]{\oldsubparagraph{#1}\mbox{}}
\fi

\begin{document}


\title{Lattice Boltzmann Simulation With Heat Transfer}

\author{CFD-Lab\\Group H} \date{Summer Term 2018}

\maketitle

\textbf{Introduction}

The code consists of the implementation of the Lattice Boltzmann Method
for 3D simulations of weakly compressible flows. LBM has its origins in
statistical mechanics. In this approach, one solves the Boltzmann
equation for the particle distribution function. This function describes
the probability of finding fluid molecules in a specific part of space,
and having a specific velocity. Then, the macroscopic variables such as
velocity and pressure are obtained by evaluating the hydrodynamic
moments of the particle distribution function. This is done by
integrating the distribution over the velocity space.

The Boltzmann equation considers a collision operator, which is a second
order differential operator representing the effect of intermolecular
collisions within the fluid. This collision operator has been linearized
with the BGK approximation.

The D3Q19 model for the velocity vectors in the 3D space.

The code offers the possibility to run two different scenarios. The
scenarios are: Driven Cavity and Natural Convection.

The Driven Cavity implementation followed the procedure in the Worksheet
2 from previous terms in the CFD-Lab course.

The implementation of Heat Transfer in the simulation for the Natural
Convection scenario followed the procedure proposed by Chih-Hao et al.
in ``Thermal boundary conditions for thermal lattice Boltzmann
simulations''.

In this approach one considers additionally from the particle density
distribution function, a particle energy distribution function, which
has the same D3Q19 velocity scheme. This distribution follows the same
streaming an collision steps as the particle density distribution
function. The computation of the temperature can be obtained in a
similar form as the density, with discrete analogs of integral
expressions of the particle energy distribution function.

The geometry of the domain and the information about boundary types is
stored in a Flag vector. The initialization of the Flag vector for each
scenario is completed by a specialized function. - The Driven Cavity
scenario has one moving wall in the upper z-face, free-slip conditions
in the y-faces and no-slip conditions on the x-faces and the bottom
z-face. - The velocity boundary conditions are free-slip in the y-faces
and no-slip on the rest. The temperature boundary conditions are
Dirichlet to the x-faces and adiabatic for the rest.

In both cases the velocity boundary conditions for the y-faces are set
to free-slip to be able to compare with 2D simulations on the x-z plane.

The two scenarios which can be parametrized by modification of .dat
files. The addition of Heat Transfer introduced a new dimensionless
relaxation time for the particle energy distribution function, which
controls the rates of approaching the equilibrium of this distribution.
To differentiate between the relaxation times we denote as `tau\_f' the
one for density distribution and `tau\_g' for energy distribution.

The sign of the gravity value in the .dat files is positive because it
contains the direction of the buoyancy force applied to the particles.
The buoyancy terms are modeled with the Boussinesq approximation,
assuming linear dependency on the temperature given by the thermal
expansion coefficient noted by `beta'.

All the model implementation was developed from the assumption for the
grid and stepsizes both being equal to one. The length chosen in the
.dat files for the domain will correspond to the number of cells used.



\textbf{Code Usage}

\begin{itemize}
\tightlist
\item
  Building the code with the Makefile creates the lbsim executable.
\item
  There are two possible scenarios:

  \begin{itemize}
  \tightlist
  \item
    Driven Cavity:

    \begin{itemize}
    \tightlist
    \item
      One can modify its parameters in cavity.dat.
    \item
      To run: ./lbsim cavity.dat
    \end{itemize}
  \item
    Natural Convection:

    \begin{itemize}
    \tightlist
    \item
      One can modify its parameters in convection.dat.
    \item
      To run: ./lbsim convection.dat
    \end{itemize}
  \end{itemize}
\end{itemize}



\textbf{OUTPUT FOLDERS}

\begin{itemize}
\tightlist
\item
  Each simulation is creating an output folder for .vtk files.
\item
  The name of the folder is the name of the scenario.
\item
  The output folder do NOT have to be cleaned, removed or manually
  created before any simulation.
\end{itemize}




\end{document}
